\section*{clear mate}

\fancyhead[RE,LO]{clear mate}

Часто мне приходилось видеть, как ребята, недавно научившиеся играть в шахматы, играют друг с другом. На доске у одного из соперников остаётся голый король, у второго - полкомплекта фигур. И начинается беготня шахами за королём, перемежаемая воскликами:  " - Сдавайся! - Сначала поставь мат!" Нередко случается, что слабейшая сторона влезает в пат, или же сильнейшая, устав от партии, соглашается на ничью. Это связано с тем, что соперники не умеют считать варианты и, как следствие, видят мат.

Научиться ставить маты несложно. Достаточно понять, как они составляются. Одним из эффективных способов является решение одноходовых позиций, которые и предлагаются в данном задачнике. Следует напомнить, что мат является шахом, от которого нет защиты. Защитами от шаха являются:

\begin{itemize}
\item{} уход короля на свободное поле;
\item{} взятие шахующей фигуры;
\item{} перекрытие шахующей фигуры.
\end{itemize}

Последняя защита иногда невозможна, например, в случае шаха конём или в случае непосредственного контакта шахующей фигуры и короля.

В данной части сборника представляются позиции, в которых есть минимальное количество фигур, - только необходимые для матовой картины. В целях упрощения подхода к задачам во всех позициях начинают белые и ставят мат в один ход. Матовые картины в задачах являются типичными, в связи с чем ученик при их запоминании будет пытаться искать подобные решения в более сложных позициях.

% an example of a battery here

Особый интерес представляют батарейные маты. Батареей называется расположение на доске двух разноходящих фигур одного цвета, одна из которых дальнобойная (ладья, слон или ферзь), при котором вторая фигура стоит на линии действия дальнобойной фигуры (см. рисунок). Батарея является мощным средством шахматиста, которое позволяет бороться с защитами от шаха. Поэтому следуетхорошо усвоить её использование.